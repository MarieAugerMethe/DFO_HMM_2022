\documentclass{article}
\usepackage{hyperref}
\usepackage[top=2in, bottom=1.5in, left=1in, right=1in]{geometry}
\usepackage{exercise}
\usepackage{amsmath}

% Paragraph indentation and line skip
\setlength{\parindent}{0cm}
\setlength{\parskip}{3mm plus2mm minus1mm}

% For tilde
\usepackage{xspace}
\newcommand{\mytilde}{\lower.80ex\hbox{\char`\~}\xspace}

\usepackage{Sweave}
\begin{document}
\input{Tutorial_HMM-concordance}

%%%%%%%%%%%%%%%%%%%%%%%%%%%%%%%%%%%%%%%%%%%%%%%%%%%%%
% Hooks
% Hook for tilde
\begin{Schunk}
\begin{Sinput}
> library(knitr)
> hook_source = knit_hooks$get('source')
> knit_hooks$set(source = function(x, options) {
+   txt = hook_source(x, options)
+   # extend the default source hook
+   gsub('~', '\\\\mytilde', txt)
+ })
\end{Sinput}
\end{Schunk}
\begin{Schunk}
\begin{Sinput}
> opts_chunk$set(fig.width=8, fig.height=6, fig.align="center", tidy=TRUE,
+                tidy.opts=list(blank=FALSE, width.cutoff=52),
+                size="large")
\end{Sinput}
\end{Schunk}

%%%%%%%%%%%%%%%%%%%%%%%%%%%%%%%%%%%%%%%%%%%%%%%%%%%%%
\author{Marie Auger-M\'eth\'e}
\title{Tutorial - Hidden Markov Models}
\date{}
\maketitle

\large

%%%%%%%%%%%%%%%%%%%%%%%%%%%%%%%%%%%%%%%%%%%%%%%%%%%%%

\section{Hidden Markov Models: tutorial goals and set up}

The goal of this tutorial is to explore how to fit hidden Markov models (HMMs) to movement data. To do so, we will investigate a new R package, \texttt{momentuHMM}. This package builds on a slightly older package, \texttt{moveHMM}, that was developed by Th\'eo Michelot , Roland Langrock, and Toby Patterson, see associated paper: \url{https://besjournals.onlinelibrary.wiley.com/doi/full/10.1111/2041-210X.12578}. \texttt{momentuHMM}, was developed by Brett McClintock and Th\'eo Michelot. \texttt{momentuHMM} has new features such as allowing for more data streams, inclusion of covariates as raster layers, and much more, see associated paper: \url{https://besjournals.onlinelibrary.wiley.com/doi/abs/10.1111/2041-210X.12995}.

% My papers: ?

\subsection{Setup and data preparation}

First, let's load the packages that we will need to complete the analyses. Off course you need to have them installed first.

\begin{Schunk}
\begin{Sinput}
> library(momentuHMM) # Package for fitting HMMs, builds on moveHMM
> library(sp) # For GIS tasks (e.g. geographic transformations)
> library(raster) # For raster spatial covariates
\end{Sinput}
\end{Schunk}

One of the main features of the data used with HMMs is that locations are taken at regular time steps and that there are negligible position error. So for example, HMMs are adequate to use on GPS tags that take locations at a set temporal frequency (e.g. every 2 hrs). Without reprocessing, HMMs are not particularly good for irregular times series of locations or locations with large measurement error (e.g. you would need to preprocess Argos data before applying HMMs).

I work mainly on aquatic species, and unfortunately, movement of aquatic species is rarely taken at regular time intervals and without measurement error. For example, the data set we will work on, is the movement track of a grey seal tagged with a Fastlock GPS tag. This is a subset of the data set used in the paper from Whoriskey et al. 2017 (\url{ https://onlinelibrary.wiley.com/doi/abs/10.1002/ece3.2795}), which applies a set of different HMMs to the movement data of various aquatic species. The original dataset is available in the online supplementary information.

Let's read the data file and peak at it.

\begin{Schunk}
\begin{Sinput}
> # Make sure you are in the good directory!
> seal <- read.csv("seal.csv", stringsAsFactors = FALSE)